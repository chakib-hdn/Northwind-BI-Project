\documentclass[12pt]{article}
\usepackage[utf8]{inputenc}
\usepackage[T1]{fontenc}
\usepackage[french]{babel}
\usepackage{graphicx}
\usepackage{hyperref}
\usepackage{listings}
\usepackage{xcolor}
\usepackage{geometry}
\usepackage{float}
\usepackage{caption}
\usepackage{tcolorbox}
\usepackage{enumitem}
\usepackage{tikz}
\usetikzlibrary{shapes.geometric, arrows, positioning, calc}

\geometry{margin=2.5cm}

\title{Guide Complet : Power BI Desktop}
% =========================
% PAGE DE GARDE
% =========================
\begin{titlepage}
    \centering
    
    % Université / Faculté
    {\Large \textbf{Université des Sciences et de la Technologie Houari Boumediene (USTHB)}\par}
    \vspace{0.5cm}
    {\large Faculté d’Informatique\par}
    \vspace{0.5cm}
    {\large Master 2 Big Data\par}
    
    \vspace{2cm}
    
    % Titre principal
    {\Huge \bfseries Conception d'un entrepot de données \par}
    {\Huge \bfseries dans power bi \par}

    
    \vspace{2cm}
    
    % Module
    {\Large \textbf{Module : Business Intelligence (BI)}\par}
    
    \vspace{1.5cm}
    
    % Étudiant
    \begin{flushleft}
        \textbf{Réalisé par :} \\
        \large IHADDADENE Chakib 181831091825
    \end{flushleft}
    
    \vspace{1cm}   
    
    \vfill
    
    % Année universitaire
    {\large Année universitaire : 2024 -- 2025\par}
    
\end{titlepage}

\newpage

\begin{document}

\maketitle

\tableofcontents

\newpage

\begin{tcolorbox}[colback=blue!5!white, colframe=blue!75!black, title=�� Téléchargement Rapide]
\url{https://aka.ms/pbidesktopstore}
\end{tcolorbox}

\section{Installation de Power BI Desktop}

\subsection{Prérequis Système}

\begin{table}[H]
\centering
\begin{tabular}{|l|l|}
\hline
\textbf{Composant} & \textbf{Configuration minimale} \\
\hline
Système d'exploitation & Windows 10 ou 11 (64-bit) \\
Processeur & 1 GHz ou plus rapide \\
Mémoire RAM & 4 GB (8 GB recommandé) \\
Espace disque & 2 GB d'espace libre \\
Résolution écran & 1440x900 ou supérieure \\
\hline
\end{tabular}
\caption{Configuration système requise}
\end{table}

\subsection{Étapes d'Installation}

\subsubsection{Méthode 2 : Téléchargement Direct}
\begin{enumerate}[label=\arabic*.]
    \item Rendez-vous sur : \url{https://aka.ms/pbidesktopstore}
    \item Cliquez sur \textbf{"Télécharger gratuitement"}
    \item Exécutez le fichier \texttt{PBIDesktopSetup.exe}
    \item Suivez l'assistant d'installation
    \item Acceptez les conditions d'utilisation
\end{enumerate}
\begin{figure}[H]
    \centering
    \includegraphics[width=0.75\linewidth]{desktop-install-01.png}
    \caption{Interface d'installation de power bi desktop}
    \label{fig:pbi-install}
\end{figure}

\section{Premier Lancement et Configuration}

\subsection{Configuration Initiale}

\begin{enumerate}[label=\arabic*.]
    \item \textbf{Écran d'accueil} : Sélectionnez "Commencer"
    \item \textbf{Connexion} : Connectez-vous avec votre compte Microsoft
    \item \textbf{Thème} : Choisissez "Clair" ou "Sombre"
    \item \textbf{Paramètres régionaux} : Sélectionnez "Français (France)"
\end{enumerate}

\subsection{Interface Utilisateur}

\begin{table}[H]
\centering
\begin{tabular}{|l|l|}
\hline
\textbf{Zone} & \textbf{Fonctionnalité} \\
\hline
1. Ruban & Commandes principales (Fichier, Accueil, etc.) \\
2. Volet Visualisations & Types de graphiques disponibles \\
3. Volet Champs & Tables et colonnes de données \\
4. Zone de dessin & Construction des rapports \\
5. Volet Filtres & Application des filtres \\
6. Barre d'état & Informations et progression \\
\hline
\end{tabular}
\end{table}

\begin{figure}[H]
    \centering
    \includegraphics[width=\linewidth]{Capture d'écran 2025-12-10 194128.PNG}
    \caption{Anatomie de l'interface Power BI Desktop}
    \label{fig:pbi-interface}
\end{figure}

\section{Architecture de l'Entrepôt de Données Northwind}

\subsection{Modèle en Étoile (Star Schema)}

% === DIAGRAMME TIKZ AJOUTÉ ICI ===
\begin{figure}[H]
    \centering
    \begin{tikzpicture}[node distance=0.8cm]
    
    % Définir les styles
    \tikzset{
        box/.style={
            draw,
            rectangle,
            align=center,
            minimum height=1.2cm,
            line width=0.8pt
        },
        titlebox/.style={
            draw,
            rectangle,
            align=center,
            fill=gray!10,
            font=\bfseries,
            minimum height=0.8cm,
            line width=0.8pt
        },
        arrow/.style={
            ->,
            >=stealth,
            line width=1pt
        }
    }
    
    % Couleurs
    \definecolor{sourcecolor}{RGB}{173, 216, 230}      % bleu clair
    \definecolor{etlcolor}{RGB}{255, 228, 181}        % orange clair
    \definecolor{factcolor}{RGB}{255, 182, 193}       % rose clair
    \definecolor{dimcolor}{RGB}{144, 238, 144}        % vert clair
    \definecolor{reportcolor}{RGB}{221, 160, 221}     % violet clair
    
    % ==================== SOURCES DE DONNÉES ====================
    \node[titlebox, minimum width=6cm] (title1) at (0,8) {SOURCES DE DONNÉES};
    \node[box, fill=sourcecolor, minimum width=6cm, below=0.2 of title1] (sources) {
        \begin{minipage}{5cm}
        \centering
        \vspace{0.1cm}
        SQL Server (SSMS)\\
        Fichiers Excel (ACCESS)
        \vspace{0.1cm}
        \end{minipage}
    };
    
    % ==================== COUCHE ETL ====================
    \node[titlebox, minimum width=6cm, below=1.2 of sources] (title2) {COUCHE ETL};
    \node[box, fill=etlcolor, minimum width=6cm, below=0.2 of title2] (etl) {
        \begin{minipage}{5cm}
        \centering
        \vspace{0.1cm}
        POWER QUERY\\
        (Transformation \& Nettoyage)
        \vspace{0.1cm}
        \end{minipage}
    };
    
    % ==================== MODÈLE EN ÉTOILE ====================
    \node[titlebox, minimum width=6cm, below=1.2 of etl] (title3) {MODÈLE EN ÉTOILE};
    
    % Table de faits
    \node[box, fill=factcolor, minimum width=4cm, minimum height=1.5cm, below=0.8 of title3] (fact) {
        \textbf{TF\_Commande}\\
        \footnotesize{(Table de faits)}
    };
    
    % Dimensions
    \node[box, fill=dimcolor, minimum width=3cm, left=3.5cm of fact] (dim_temps) {Dim\_Temps};
    \node[box, fill=dimcolor, minimum width=3cm, above=0.6 of dim_temps] (dim_client) {Dim\_Client};
    \node[box, fill=dimcolor, minimum width=3cm, below=0.6 of dim_temps] (dim_employee) {Dim\_Employee};
    
    % ==================== COUCHE RAPPORT ====================
    \node[titlebox, minimum width=6cm, below=3 of fact] (title4) {COUCHE RAPPORT};
    \node[box, fill=reportcolor, minimum width=6cm, minimum height=1.5cm, below=0.2 of title4] (report) {
        \textbf{POWER BI DASHBOARD}\\
        KPI + Graphiques + Tableaux
    };
    
    % ==================== FLÈCHES ====================
    % Sources → ETL
    \draw[arrow, color=gray!70] (sources.south) -- (title2.north);
    
    % ETL → Modèle
    \draw[arrow, color=gray!70] (etl.south) -- (title3.north);
    
    % Dimensions → Table de faits
    \draw[arrow, color=gray!70] (dim_temps.east) -- (fact.west);
    \draw[arrow, color=gray!70] (dim_client.east) -- ($(fact.west) + (0,0.3)$);
    \draw[arrow, color=gray!70] (dim_employee.east) -- ($(fact.west) + (0,-0.3)$);
    
    % Table de faits → Rapport
    \draw[arrow, color=gray!70] (fact.south) -- (title4.north);
    
    % Titre → Contenu (lignes verticales internes)
    \draw[gray!50, line width=0.5pt] (title1.south) -- (sources.north);
    \draw[gray!50, line width=0.5pt] (title2.south) -- (etl.north);
    \draw[gray!50, line width=0.5pt] (title3.south) -- (fact.north);
    \draw[gray!50, line width=0.5pt] (title4.south) -- (report.north);
    
    % Cadre autour du modèle en étoile
    \draw[gray!60, line width=0.8pt, rounded corners=5pt] 
        ($(title3.north west) + (-0.5,0.3)$) rectangle 
        ($(dim_employee.south east) + (0.5,-0.3)$);
    
    \end{tikzpicture}
    \caption{Architecture complète de l'entrepôt de données Northwind avec modèle en étoile à 3 dimensions}
    \label{fig:architecture-northwind}
\end{figure}
% === FIN DU DIAGRAMME TIKZ ===

\subsection{Composants de l'Architecture}

\begin{table}[H]
\centering
\begin{tabular}{|p{0.3\textwidth}|p{0.6\textwidth}|}
\hline
\textbf{Composant} & \textbf{Description} \\
\hline
Sources de Données & SQL Server (Northwind) + Fichiers Excel \\
Couche ETL & Power Query pour l'extraction et transformation \\
Tables Dimensions & Dim\_Employee, Dim\_Client, Dim\_Temps \\
Table de Faits & TF\_Commande  \\
Couche Présentation & Rapports Power BI + Dashboard \\
\hline
\end{tabular}
\caption{Composants de l'architecture de l'entrepôt}
\end{table}

\subsection{Flux de Données}

\begin{enumerate}[label=\arabic*.]
    \item \textbf{Extraction} : Données brutes depuis SSMS et Excel
    \item \textbf{Transformation} : Nettoyage et standardisation via Power Query
    \item \textbf{Chargement} : Création des dimensions et table de faits
    \item \textbf{Modélisation} : Établissement des relations 1:*
    \item \textbf{Visualisation} : Création des rapports et dashboards
\end{enumerate}

\section{Création du Projet Northwind}

\subsection{Structure du Projet}

\begin{enumerate}[label=\arabic*.]
    \item \textbf{Créer un dossier projet} : 
    \begin{lstlisting}[language=bash]
C:\Users\IHADDADENE Chakib\Documents\Power BI Projects\Northwind_DW
    └── data\          # Données sources
    └── scripts\       # Codes Power Query
    └── exports\       # Exports et rapports
    \end{lstlisting}
    
    \item \textbf{Lancer Power BI Desktop}
    \item \textbf{Nouveau fichier} : Fichier → Nouveau
    \item \textbf{Enregistrer} : Fichier → Enregistrer sous → "Northwind\_DW.pbix"
\end{enumerate}

\subsection{Importation des Données Sources}

\subsubsection{Connexion à SQL Server}
\begin{enumerate}[label=\arabic*.]
    \item Cliquez sur \textbf{"Obtenir des données"} → \textbf{Base de données} → \textbf{SQL Server}
    \item Entrez les informations de connexion :
    \begin{itemize}
        \item Serveur : \texttt{DESKTOP-VEO1CEQ\textbackslash SQLCHAKIB}
        \item Base de données : \texttt{Northwind}
        \item Mode de connexion : \textbf{Importer} (recommandé)
    \end{itemize}
    \item Sélectionnez les tables nécessaires :
    \begin{itemize}
        \item Customers
        \item Employees
        \item EmployeeTerritories
        \item Orders
        \item Territories
    \end{itemize}
\end{enumerate}

\subsubsection{Importation des Fichiers Excel}
\begin{enumerate}[label=\arabic*.]
    \item Cliquez sur \textbf{"Obtenir des données"} → \textbf{Fichier} → \textbf{Excel}
    \item Naviguez vers : \texttt{C:\textbackslash Users\textbackslash PC\textbackslash Documents\textbackslash M2 BIGDATA\textbackslash tp bi\textbackslash csv du dm}
    \item Sélectionnez les fichiers :
    \begin{itemize}
        \item Customers.xlsx
        \item Employees2.xlsx
        \item Orders.xlsx
    \end{itemize}
\end{enumerate}

\subsection{Transformation avec Power Query}

\subsubsection{Accès à l'Éditeur Power Query}
\begin{itemize}
    \item Méthode 1 : \textbf{Accueil} → \textbf{Transformer les données}
    \item Méthode 2 : Clic droit sur une requête → \textbf{Éditer la requête}
\end{itemize}

\begin{figure}[H]
    \centering
    \includegraphics[width=1\linewidth]{Capture d'écran 2025-12-10 094846.png}
    \caption{Éditeur Power Query avec requêtes Northwind
}
    \label{fig:placeholder}
\end{figure}
\subsubsection{Renommage des Requêtes}
\begin{enumerate}[label=\arabic*.]
    \item Dans le volet de navigation Power Query :
    \item Renommez chaque requête avec le suffixe \texttt{\_ssms} ou \texttt{\_excel} :
    \begin{itemize}
        \item Customers → Customers\_ssms
        \item Employees → Employee\_ssms
        \item etc.
    \end{itemize}
\end{enumerate}

\section{Implémentation du Modèle de Données}

\subsection{Création des Dimensions}

\subsubsection{Dim\_Employee}
\begin{enumerate}[label=\arabic*.]
    \item \textbf{Nouvelle requête} : Accueil → Nouvelle source → Requête vide
    \item \textbf{Coller le code} : Collez le code M de Dim\_Employee
    \item \textbf{Renommer} : Renommez la requête en \texttt{Dim\_Employee}
    \item \textbf{Fermer et appliquer}
\end{enumerate}

\subsubsection{Dim\_Client et Dim\_Temps}
\begin{itemize}
    \item Répétez le processus pour chaque dimension
    \item Vérifiez les types de données
    \item Activez le chargement pour les tables dimensionnelles
\end{itemize}

\subsection{Création de la Table de Faits}

\subsubsection{TF\_Commande}
\begin{enumerate}[label=\arabic*.]
    \item Créez une nouvelle requête vide
    \item Collez le code complet de TF\_Commande
    \item Renommez en \texttt{TF\_Commande}
    \item Vérifiez le nombre de lignes (devrait être 878)
\end{enumerate}

\subsection{Établissement des Relations}

\begin{enumerate}[label=\arabic*.]
    \item Retournez dans la vue \textbf{Modèle} de Power BI
    \item Établissez les relations suivantes :
    \begin{itemize}
        \item \texttt{TF\_Commande[id\_temps] → Dim\_Temps[id\_temps]}
        \item \texttt{TF\_Commande[id\_seqEmployee] → Dim\_Employee[id\_seqEmployee]}
        \item \texttt{TF\_Commande[id\_seqClient] → Dim\_Client[id\_seqClient]}
    \end{itemize}
    \item Vérifiez que toutes les relations sont \textbf{1 à plusieurs (*)}
\end{enumerate}

\begin{figure}[H]
    \centering
    \includegraphics[width=1\linewidth]{Capture d'écran 2025-12-08 221202.png}
    \caption{Modèle de données final avec relations}
    \label{fig:placeholder}
\end{figure}

\section{Comparaison Power BI vs Talend pour l'ETL}

\subsection{Tableau Comparatif}

\begin{table}[H]
\centering
\begin{tabular}{|p{0.25\textwidth}|p{0.35\textwidth}|p{0.35\textwidth}|}
\hline
\textbf{Aspect} & \textbf{Power BI (Power Query)} & \textbf{Talend} \\
\hline
\textbf{Type d'outil} & Outil de BI avec ETL intégré & Plateforme ETL/ELT dédiée \\
\hline
\textbf{Complexité} & Courbe d'apprentissage douce & Courbe d'apprentissage plus raide \\
\hline
\textbf{Langage} & M (Power Query) & Java + Composants visuels \\
\hline
\textbf{Connexions} & Connecteurs natifs limités & 1000+ connecteurs \\
\hline
\textbf{Transformation} & Interface utilisateur + M & Interface graphique + code \\
\hline
\textbf{Orchestration} & Basic (rafraîchissements) & Avancée (workflows complexes) \\
\hline
\textbf{Coût} & Gratuit (Desktop) & Licence payante (Enterprise) \\
\hline
\textbf{Performance} & Optimisé pour données de taille moyenne & Scalable pour gros volumes \\
\hline
\textbf{Maintenance} & Facile (tout intégré) & Complexe (infrastructure séparée) \\
\hline
\textbf{Utilisation projet Northwind} & Parfait pour POC et démonstrations & Surdimensionné pour ce besoin \\
\hline
\end{tabular}
\caption{Comparaison Power BI vs Talend pour l'ETL}
\end{table}

\subsection{Points Forts par Outil}

\subsubsection{Power BI (Power Query)}
\begin{itemize}
    \item ✅ Intégration native avec les rapports
    \item ✅ Interface utilisateur intuitive
    \item ✅ Pas de contexte switching entre ETL et reporting
    \item ✅ Version Desktop gratuite
    \item ✅ Fonctionnalités de modélisation intégrées
    \item ✅ Support DAX pour calculs avancés
\end{itemize}

\subsubsection{Talend}
\begin{itemize}
    \item ✅ Orchestration de workflows complexes
    \item ✅ Support de gros volumes de données
    \item ✅ Connecteurs très variés
    \item ✅ Qualité et gouvernance des données
    \item ✅ Métadonnées et documentation
    \item ✅ Intégration avec l'écosystème Big Data
\end{itemize}

\subsection{Exemple de Code Comparé}

\begin{table}[H]
\centering
\begin{tabular}{|p{0.45\textwidth}|p{0.45\textwidth}|}
\hline
\textbf{Power Query (M)} & \textbf{Talend (Java)} \\
\hline
\begin{lstlisting}[language=PowerQuery, basicstyle=\tiny]
let
    Source = Orders_ssms,
    #"Filtered Rows" = 
        Table.SelectRows(
            Source, 
            each [ShippedDate] <> null
        )
in
    #"Filtered Rows"
\end{lstlisting} &
\begin{lstlisting}[language=Java, basicstyle=\tiny]
// Composant tFilterRow
row1.ShippedDate != null

// Ou en Java dans tJavaRow
if(input_row.ShippedDate != null) {
    output_row = input_row;
}
\end{lstlisting} \\
\hline
\end{tabular}
\caption{Comparaison de syntaxe pour un filtre simple}
\end{table}

\subsection{Recommandation pour le Projet Northwind}

\begin{tcolorbox}[colback=yellow!5!white, colframe=yellow!75!black, title=�� Choix Optimal pour Northwind]
\textbf{Power BI avec Power Query est le meilleur choix} pour ce projet car :
\begin{itemize}
    \item Volume de données modeste (878 lignes)
    \item Besoin d'intégration directe avec les rapports
    \item Temps de développement réduit
    \item Aucune infrastructure supplémentaire nécessaire
    \item Coût nul avec Power BI Desktop gratuit
    \item Facilité de maintenance et de partage
\end{itemize}
\end{tcolorbox}

\section{Visualisation avec Python dans Power BI}

\subsection{Configuration de Python pour Power BI}

\subsubsection{Installation des Prérequis}
\begin{enumerate}[label=\arabic*.]
    \item \textbf{Installer Python} : Téléchargez Python 3.8+ depuis \url{https://python.org}
    \item \textbf{Installer les packages} :
    \begin{lstlisting}[language=bash]
pip install pandas matplotlib
    \end{lstlisting}
    \item \textbf{Configurer Power BI} :
    \begin{itemize}
        \item Fichier → Options et paramètres → Options
        \item Scripting Python → Spécifier le chemin d'installation Python
        \item Exemple : \texttt{C:\textbackslash Users\textbackslash IHADDADENE Chakib\textbackslash AppData\textbackslash Local\textbackslash Programs\textbackslash Python\textbackslash Python39}
    \end{itemize}
\end{enumerate}


\subsubsection{Création d'un Visuel Python}
\begin{enumerate}[label=\arabic*.]
    \item \textbf{Sélectionner le visuel Python} :
    \begin{itemize}
        \item Dans le volet Visualisations, cliquez sur l'icône \textbf{Python}
        \item Une zone de script vide apparaît sur la page
    \end{itemize}
    
    \item \textbf{Glisser les champs nécessaires} :
    \begin{itemize}
        \item Depuis le volet Champs, glissez-déposez les colonnes dans le visuel Python
        \item Exemple pour notre projet Northwind :
        \begin{itemize}
            
            \item Depuis \texttt{TF\_Commande} : \texttt{nbr\_commande\_livrees}
            \item Depuis \texttt{Dim\_Employee} : \texttt{TerritoryDescri}
        \end{itemize}
    \end{itemize}
    
    \item \textbf{Écrire le script Python} :
    \begin{itemize}
        \item Le champ de script s'active automatiquement
        \item Power BI crée automatiquement un DataFrame \texttt{dataset} contenant vos données
    \end{itemize}
\end{enumerate}

\begin{figure}[H]
    \centering
    \includegraphics[width=1\linewidth]{script_python.png}
    \caption{Interface du visuel Python dans Power BI
}
    \label{fig:placeholder}
\end{figure}
\subsubsection{Exemple Complet : Analyse des Commandes livrées par territoire }
\subsection{Exemple 2 : Top 5 des Territoires par Performance}

\subsubsection{Préparation des Données}
\begin{enumerate}[label=\arabic*.]
    \item \textbf{Sélectionner le visuel Python} dans le volet Visualisations
    \item \textbf{Glisser les champs nécessaires} :
    \begin{itemize}
        \item Depuis \texttt{Dim\_Employee} : \texttt{TerritoryDescri}
        \item Depuis \texttt{TF\_Commande} : \texttt{nbr\_commande\_livrees}
    \end{itemize}
    \item \textbf{Power BI crée automatiquement} le DataFrame \texttt{dataset}
\end{enumerate}

\subsubsection{Script Python pour l'Analyse Territoriale}
\begin{lstlisting}
import pandas as pd
import matplotlib.pyplot as plt

df = dataset.groupby('TerritoryDescri')['nbr_commande_livrees'] \
            .sum() \
            .nlargest(5)

plt.figure(figsize=(8,6))
df.sort_values().plot(kind='barh', color='purple')
plt.title("Top 5 Territoires par commandes livrees")
plt.xlabel("Nombre de commandes")
plt.tight_layout()
plt.show()
\end{lstlisting}

\subsubsection{Résultat Attendue}
\begin{figure}[H]
    \centering
    \includegraphics[width=0.75\linewidth]{graphe.png}
    \caption{Visualisation du top 5 des territoires générée par Python}
    \label{fig:placeholder}
\end{figure}

\subsubsection{Explications du Code}
\begin{table}[H]
\centering
\begin{tabular}{|p{0.25\textwidth}|p{0.7\textwidth}|}
\hline
\textbf{Ligne de code} & \textbf{Explication} \\
\hline
\texttt{dataset.groupby()} & Regroupe les données par territoire \\
\texttt{nlargest(5)} & Sélectionne les 5 territoires avec le plus de commandes \\
\texttt{kind='barh'} & Crée un diagramme en barres horizontal \\
\texttt{color='purple'} & Définit la couleur des barres \\
\texttt{plt.title()} & Ajoute un titre au graphique \\
\texttt{plt.text()} & Ajoute les valeurs numériques sur les barres \\
\texttt{plt.grid()} & Ajoute une grille pour meilleure lisibilité \\
\texttt{plt.tight\_layout()} & Optimise l'espacement des éléments \\
\texttt{plt.show()} & Affiche le graphique dans Power BI \\
\hline
\end{tabular}
\caption{Explications détaillées du code Python}
\end{table}
\subsubsection{Exécution et Résultats}
\begin{enumerate}[label=\arabic*.]
    \item \textbf{Cliquez sur "Exécuter le script"} (icône play)
    \item \textbf{Attendez le chargement} (première exécution peut prendre quelques secondes)
    \item \textbf{Visualisez le résultat} : Le graphique apparaît dans le visuel
    \item \textbf{Interagissez} : Utilisez les filtres Power BI pour mettre à jour dynamiquement le graphique Python
\end{enumerate}

\subsection{Bonnes Pratiques pour les Visuals Python}

\begin{table}[H]
\centering
\begin{tabular}{|p{0.4\textwidth}|p{0.55\textwidth}|}
\hline
\textbf{Pratique} & \textbf{Explication} \\
\hline
Glisser d'abord les champs & Power BI crée \texttt{dataset} avec seulement ces colonnes \\
Vérifier les colonnes & Utilisez \texttt{print(dataset.columns)} pour debugger \\
Gérer les NaN & Utilisez \texttt{dropna()} ou \texttt{fillna()} \\
Optimiser les performances & Limitez le nombre de lignes avec des filtres \\
Utiliser des styles & \texttt{plt.style.use('seaborn')} pour de beaux graphiques \\
Sauvegarder les scripts & Copiez vos scripts dans un fichier .py externe \\
\hline
\end{tabular}
\caption{Bonnes pratiques pour les visuals Python}
\end{table}

\subsection{Avantages du Visuel Python}

\begin{itemize}
    \item \textbf{✅ Intégration native} : Pas besoin d'exporter/importer
    \item \textbf{✅ Interactivité} : Réagit aux filtres Power BI
    \item \textbf{✅ Flexibilité} : Toute bibliothèque Python compatible
    \item \textbf{✅ Mise à jour automatique} : Rafraîchit avec les données
\end{itemize}

\begin{tcolorbox}[colback=yellow!5!white, colframe=yellow!75!black, title=⚠️ Limitations à connaître]
\begin{itemize}
    \item Performance : Les scripts longs peuvent ralentir Power BI
    \item Mémoire : Limité par les ressources de Power BI
    \item Dépendances : Doivent être installées sur chaque machine
    \item Version Python : Doit correspondre à celle configurée
\end{itemize}
\end{tcolorbox}
\newpage
\subsection{Dashboard Final Northwind}


Notre dashboard final intègre 6 visuals Python , les voici :

\begin{figure}[H]
    \centering
    \includegraphics[width=1\linewidth]{Capture d'écran 2025-12-13 111445.png}
    \caption{Dashboard Northwind avec 6 visuals Python}
    \label{fig:placeholder}
\end{figure}


\section{Résolution des Problèmes Courants}

\subsection{Problèmes d'Installation}

\begin{table}[H]
\centering
\begin{tabular}{|p{0.3\textwidth}|p{0.6\textwidth}|}
\hline
\textbf{Problème} & \textbf{Solution} \\
\hline
Échec d'installation Microsoft Store & Téléchargez la version hors Store \\
Erreur "Accès refusé" & Exécutez en tant qu'administrateur \\
Manque de dépendances & Installez .NET Framework 4.8 \\
Antivirus bloque l'installation & Désactivez temporairement l'antivirus \\
\hline
\end{tabular}
\end{table}

\subsection{Problèmes de Connexion}

\begin{table}[H]
\centering
\begin{tabular}{|p{0.3\textwidth}|p{0.6\textwidth}|}
\hline
\textbf{Problème} & \textbf{Solution} \\
\hline
SQL Server inaccessible & Vérifiez le nom du serveur et les permissions \\
Fichier Excel non trouvé & Vérifiez le chemin et les permissions \\
Erreur d'authentification & Utilisez l'authentification Windows \\
Limite de mémoire & Augmentez la mémoire allouée à Power BI \\
\hline
\end{tabular}
\end{table}

\newpage
\section{Conclusion}

Ce guide vous a accompagné de l'installation de Power BI Desktop à la création d'un projet d'entrepôt de données complet avec Northwind, incluant une analyse comparative avec Talend. Vous disposez maintenant de toutes les ressources nécessaires pour :

\begin{itemize}
    \item ✅ Installer et configurer Power BI Desktop
    \item ✅ Comprendre l'architecture d'un entrepôt de données
    \item ✅ Comparer Power BI et Talend pour l'ETL
    \item ✅ Créer un projet structuré Northwind
    \item ✅ Importer et transformer des données multi-sources
    \item ✅ Construire un modèle de données optimisé
    \item ✅ Publier et partager vos analyses
\end{itemize}

\end{document}