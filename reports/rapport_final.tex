\documentclass[12pt]{article}
\usepackage[utf8]{inputenc}
\usepackage[T1]{fontenc}
\usepackage[french]{babel}
\usepackage{graphicx}
\usepackage{hyperref}
\usepackage{listings}
\usepackage{xcolor}
\usepackage{geometry}
\usepackage{float}
\usepackage{caption}
\usepackage{tcolorbox}
\usepackage{enumitem}
\usepackage{tikz}
\usetikzlibrary{shapes.geometric, arrows, positioning, calc}

\geometry{margin=2.5cm}

% Configuration pour les listings avec fond noir
\definecolor{darkbg}{RGB}{40, 40, 40}
\definecolor{codegreen}{RGB}{152, 195, 121}
\definecolor{codegray}{RGB}{171, 178, 191}
\definecolor{codepurple}{RGB}{198, 120, 221}

\lstdefinestyle{mystyle}{
    backgroundcolor=\color{darkbg},
    commentstyle=\color{codegreen},
    keywordstyle=\color{codepurple},
    numberstyle=\tiny\color{codegray},
    stringstyle=\color{codegreen},
    basicstyle=\ttfamily\footnotesize\color{white},
    breakatwhitespace=false,
    breaklines=true,
    captionpos=b,
    keepspaces=true,
    numbers=left,
    numbersep=5pt,
    showspaces=false,
    showstringspaces=false,
    showtabs=false,
    tabsize=2,
    frame=single,
    framerule=0.5pt,
    rulecolor=\color{gray!50}
}

\lstset{style=mystyle}

\title{Guide Complet : Power BI Desktop}
% =========================
% PAGE DE GARDE
% =========================
\begin{titlepage}
    \centering
    
    % Université / Faculté
    {\Large \textbf{Université des Sciences et de la Technologie Houari Boumediene (USTHB)}\par}
    \vspace{0.5cm}
    {\large Faculté d'Informatique\par}
    \vspace{0.5cm}
    {\large Master 2 Big Data\par}
    
    \vspace{2cm}
    
    % Titre principal
    {\Huge \bfseries Conception d'un entrepot de données \par}
    {\Huge \bfseries dans power bi \par}

    
    \vspace{2cm}
    
    % Module
    {\Large \textbf{Module : Business Intelligence (BI)}\par}
    
    \vspace{1.5cm}
    
    % Étudiant
    \begin{flushleft}
        \textbf{Réalisé par :} \\
        \large IHADDADENE Chakib 181831091825
    \end{flushleft}
    
    \vspace{1cm}   
    
    \vfill
    
    % Année universitaire
    {\large Année universitaire : 2024 -- 2025\par}
    
\end{titlepage}

\newpage

\begin{document}

\maketitle

\tableofcontents

\newpage

\begin{tcolorbox}[colback=blue!5!white, colframe=blue!75!black, title=⚠️ Téléchargement Rapide]
\url{https://aka.ms/pbidesktopstore}
\end{tcolorbox}

\section{Installation de Power BI Desktop}

\subsection{Prérequis Système}

\begin{table}[H]
\centering
\begin{tabular}{|l|l|}
\hline
\textbf{Composant} & \textbf{Configuration minimale} \\
\hline
Système d'exploitation & Windows 10 ou 11 (64-bit) \\
Processeur & 1 GHz ou plus rapide \\
Mémoire RAM & 4 GB (8 GB recommandé) \\
Espace disque & 2 GB d'espace libre \\
Résolution écran & 1440x900 ou supérieure \\
\hline
\end{tabular}
\caption{Configuration système requise}
\end{table}

\subsection{Étapes d'Installation}

\subsubsection{Méthode 2 : Téléchargement Direct}
\begin{enumerate}[label=\arabic*.]
    \item Rendez-vous sur : \url{https://aka.ms/pbidesktopstore}
    \item Cliquez sur \textbf{"Télécharger gratuitement"}
    \item Exécutez le fichier \texttt{PBIDesktopSetup.exe}
    \item Suivez l'assistant d'installation
    \item Acceptez les conditions d'utilisation
\end{enumerate}
\begin{figure}[H]
    \centering
    \includegraphics[width=0.75\linewidth]{Interface d'installation de power bi desktop.png}
    \caption{Interface d'installation de power bi desktop}
    \label{fig:pbi-install}
\end{figure}

\section{Premier Lancement et Configuration}

\subsection{Configuration Initiale}

\begin{enumerate}[label=\arabic*.]
    \item \textbf{Écran d'accueil} : Sélectionnez "Commencer"
    \item \textbf{Connexion} : Connectez-vous avec votre compte Microsoft
    \item \textbf{Thème} : Choisissez "Clair" ou "Sombre"
    \item \textbf{Paramètres régionaux} : Sélectionnez "Français (France)"
\end{enumerate}

\subsection{Interface Utilisateur}

\begin{table}[H]
\centering
\begin{tabular}{|l|l|}
\hline
\textbf{Zone} & \textbf{Fonctionnalité} \\
\hline
1. Ruban & Commandes principales (Fichier, Accueil, etc.) \\
2. Volet Visualisations & Types de graphiques disponibles \\
3. Volet Champs & Tables et colonnes de données \\
4. Zone de dessin & Construction des rapports \\
5. Volet Filtres & Application des filtres \\
6. Barre d'état & Informations et progression \\
\hline
\end{tabular}
\end{table}

\begin{figure}[H]
    \centering
    \includegraphics[width=\linewidth]{Anatomie de l'interface Power BI Desktop.PNG}
    \caption{Anatomie de l'interface Power BI Desktop}
    \label{fig:pbi-interface}
\end{figure}

\section{Architecture de l'Entrepôt de Données Northwind}

\subsection{Modèle en Étoile (Star Schema)}

% === DIAGRAMME TIKZ AJOUTÉ ICI ===
\begin{figure}[H]
    \centering
    \begin{tikzpicture}[node distance=0.8cm]
    
    % Définir les styles
    \tikzset{
        box/.style={
            draw,
            rectangle,
            align=center,
            minimum height=1.2cm,
            line width=0.8pt
        },
        titlebox/.style={
            draw,
            rectangle,
            align=center,
            fill=gray!10,
            font=\bfseries,
            minimum height=0.8cm,
            line width=0.8pt
        },
        arrow/.style={
            ->,
            >=stealth,
            line width=1pt
        }
    }
    
    % Couleurs
    \definecolor{sourcecolor}{RGB}{173, 216, 230}      % bleu clair
    \definecolor{etlcolor}{RGB}{255, 228, 181}        % orange clair
    \definecolor{factcolor}{RGB}{255, 182, 193}       % rose clair
    \definecolor{dimcolor}{RGB}{144, 238, 144}        % vert clair
    \definecolor{reportcolor}{RGB}{221, 160, 221}     % violet clair
    
    % ==================== SOURCES DE DONNÉES ====================
    \node[titlebox, minimum width=6cm] (title1) at (0,8) {SOURCES DE DONNÉES};
    \node[box, fill=sourcecolor, minimum width=6cm, below=0.2 of title1] (sources) {
        \begin{minipage}{5cm}
        \centering
        \vspace{0.1cm}
        SQL Server (SSMS)\\
        Fichiers Excel (ACCESS)
        \vspace{0.1cm}
        \end{minipage}
    };
    
    % ==================== COUCHE ETL ====================
    \node[titlebox, minimum width=6cm, below=1.2 of sources] (title2) {COUCHE ETL};
    \node[box, fill=etlcolor, minimum width=6cm, below=0.2 of title2] (etl) {
        \begin{minipage}{5cm}
        \centering
        \vspace{0.1cm}
        POWER QUERY\\
        (Transformation \& Nettoyage)
        \vspace{0.1cm}
        \end{minipage}
    };
    
    % ==================== MODÈLE EN ÉTOILE ====================
    \node[titlebox, minimum width=6cm, below=1.2 of etl] (title3) {MODÈLE EN ÉTOILE};
    
    % Table de faits
    \node[box, fill=factcolor, minimum width=4cm, minimum height=1.5cm, below=0.8 of title3] (fact) {
        \textbf{TF\_Commande}\\
        \footnotesize{(Table de faits)}
    };
    
    % Dimensions
    \node[box, fill=dimcolor, minimum width=3cm, left=3.5cm of fact] (dim_temps) {Dim\_Temps};
    \node[box, fill=dimcolor, minimum width=3cm, above=0.6 of dim_temps] (dim_client) {Dim\_Client};
    \node[box, fill=dimcolor, minimum width=3cm, below=0.6 of dim_temps] (dim_employee) {Dim\_Employee};
    
    % ==================== COUCHE RAPPORT ====================
    \node[titlebox, minimum width=6cm, below=3 of fact] (title4) {COUCHE RAPPORT};
    \node[box, fill=reportcolor, minimum width=6cm, minimum height=1.5cm, below=0.2 of title4] (report) {
        \textbf{POWER BI DASHBOARD}\\
        KPI + Graphiques + Tableaux
    };
    
    % ==================== FLÈCHES ====================
    % Sources → ETL
    \draw[arrow, color=gray!70] (sources.south) -- (title2.north);
    
    % ETL → Modèle
    \draw[arrow, color=gray!70] (etl.south) -- (title3.north);
    
    % Dimensions → Table de faits
    \draw[arrow, color=gray!70] (dim_temps.east) -- (fact.west);
    \draw[arrow, color=gray!70] (dim_client.east) -- ($(fact.west) + (0,0.3)$);
    \draw[arrow, color=gray!70] (dim_employee.east) -- ($(fact.west) + (0,-0.3)$);
    
    % Table de faits → Rapport
    \draw[arrow, color=gray!70] (fact.south) -- (title4.north);
    
    % Titre → Contenu (lignes verticales internes)
    \draw[gray!50, line width=0.5pt] (title1.south) -- (sources.north);
    \draw[gray!50, line width=0.5pt] (title2.south) -- (etl.north);
    \draw[gray!50, line width=0.5pt] (title3.south) -- (fact.north);
    \draw[gray!50, line width=0.5pt] (title4.south) -- (report.north);
    
    % Cadre autour du modèle en étoile
    \draw[gray!60, line width=0.8pt, rounded corners=5pt] 
        ($(title3.north west) + (-0.5,0.3)$) rectangle 
        ($(dim_employee.south east) + (0.5,-0.3)$);
    
    \end{tikzpicture}
    \caption{Architecture complète de l'entrepôt de données Northwind avec modèle en étoile à 3 dimensions}
    \label{fig:architecture-northwind}
\end{figure}
% === FIN DU DIAGRAMME TIKZ ===

\subsection{Composants de l'Architecture}

\begin{table}[H]
\centering
\begin{tabular}{|p{0.3\textwidth}|p{0.6\textwidth}|}
\hline
\textbf{Composant} & \textbf{Description} \\
\hline
Sources de Données & SQL Server (Northwind) + Fichiers Excel \\
Couche ETL & Power Query pour l'extraction et transformation \\
Tables Dimensions & Dim\_Employee, Dim\_Client, Dim\_Temps \\
Table de Faits & TF\_Commande  \\
Couche Présentation & Rapports Power BI + Dashboard \\
\hline
\end{tabular}
\caption{Composants de l'architecture de l'entrepôt}
\end{table}

\subsection{Flux de Données}

\begin{enumerate}[label=\arabic*.]
    \item \textbf{Extraction} : Données brutes depuis SSMS et Excel
    \item \textbf{Transformation} : Nettoyage et standardisation via Power Query
    \item \textbf{Chargement} : Création des dimensions et table de faits
    \item \textbf{Modélisation} : Établissement des relations 1:*
    \item \textbf{Visualisation} : Création des rapports et dashboards
\end{enumerate}

\section{Création du Projet Northwind}

\subsection{Structure du Projet}

\begin{enumerate}[label=\arabic*.]
    \item \textbf{Créer un dossier projet} : 
    \begin{lstlisting}[language=bash]
C:\Users\IHADDADENE Chakib\Documents\Power BI Projects\Northwind_DW
    └── data\          # Données sources
    └── scripts\       # Codes Power Query
    └── exports\       # Exports et rapports
    \end{lstlisting}
    
    \item \textbf{Lancer Power BI Desktop}
    \item \textbf{Nouveau fichier} : Fichier → Nouveau
    \item \textbf{Enregistrer} : Fichier → Enregistrer sous → "Northwind\_DW.pbix"
\end{enumerate}

\subsection{Importation des Données Sources}

\subsubsection{Connexion à SQL Server}
\begin{enumerate}[label=\arabic*.]
    \item Cliquez sur \textbf{"Obtenir des données"} → \textbf{Base de données} → \textbf{SQL Server}
    \item Entrez les informations de connexion :
    \begin{itemize}
        \item Serveur : \texttt{DESKTOP-VEO1CEQ\textbackslash SQLCHAKIB}
        \item Base de données : \texttt{Northwind}
        \begin{figure}[H]
            \centering
            \includegraphics[width=0.75\linewidth]{connexion ssms.png}
            \caption{Connexion a la base de données SSMS}
            \label{fig:placeholder}
        \end{figure}
        \item Mode de connexion : \textbf{Importer} (recommandé)
    \end{itemize}
    \item Sélectionnez les tables nécessaires :
    \begin{itemize}
        \item Customers
        \item Employees
        \item EmployeeTerritories
        \item Orders
        \item Territories
    \end{itemize}
\end{enumerate}
\begin{figure}[H]
    \centering
    \includegraphics[width=0.75\linewidth]{selection des données ssms.png}
    \caption{Selection des tables nécessaires}
    \label{fig:placeholder}
\end{figure}

\subsubsection{Importation des Fichiers Excel}
\begin{enumerate}[label=\arabic*.]
    \item Cliquez sur \textbf{"Obtenir des données"} → \textbf{Fichier} → \textbf{Excel}
    \item Naviguez vers : \texttt{C:\textbackslash Users\textbackslash PC\textbackslash Documents\textbackslash M2 BIGDATA\textbackslash tp bi\textbackslash csv du dm}
    \item Sélectionnez les fichiers :
    \begin{itemize}
        \item Customers.xlsx
        \item Employees2.xlsx
        \item Orders.xlsx
    \end{itemize}
\end{enumerate}
\begin{figure}[H]
    \centering
    \includegraphics[width=0.5\linewidth]{Imporation des données excel.png}
    \caption{Importation des données excel}
    \label{fig:placeholder}
\end{figure}

\subsection{Transformation avec Power Query}

\subsubsection{Accès à l'Éditeur Power Query}
\begin{itemize}
    \item Méthode 1 : \textbf{Accueil} → \textbf{Transformer les données}
    \item Méthode 2 : Clic droit sur une requête → \textbf{Éditer la requête}
\end{itemize}

\begin{figure}[H]
    \centering
    \includegraphics[width=1\linewidth]{Éditeur Power Query avec requêtes Northwind.png}
    \caption{Éditeur Power Query avec requêtes Northwind}
    \label{fig:power-query-editor}
\end{figure}

\subsubsection{Renommage des Requêtes}
\begin{enumerate}[label=\arabic*.]
    \item Dans le volet de navigation Power Query :
    \item Renommez chaque requête avec le suffixe \texttt{\_ssms} ou \texttt{\_excel} :
    \begin{itemize}
        \item Customers → Customers\_ssms
        \item Employees → Employee\_ssms
        \item etc.
    \end{itemize}
\end{enumerate}

\section{Implémentation du Modèle de Données}

\subsection{Création des Dimensions}

\subsubsection{Dim\_Employee}
\begin{enumerate}[label=\arabic*.]
    \item \textbf{Nouvelle requête} : Accueil → Nouvelle source → Requête vide
    \item \textbf{Coller le code} : Collez le code M de Dim\_Employee
    \item \textbf{Renommer} : Renommez la requête en \texttt{Dim\_Employee}
    \item \textbf{Fermer et appliquer}
\end{enumerate}

\subsubsection{Dim\_Client et Dim\_Temps}
\begin{itemize}
    \item Répétez le processus pour chaque dimension
    \item Vérifiez les types de données
    \item Activez le chargement pour les tables dimensionnelles
\end{itemize}

\subsection{Création de la Table de Faits}

\subsubsection{TF\_Commande}
\begin{enumerate}[label=\arabic*.]
    \item Créez une nouvelle requête vide
    \item Collez le code complet de TF\_Commande
    \item Renommez en \texttt{TF\_Commande}
    \item Vérifiez le nombre de lignes (devrait être 878)
\end{enumerate}

\subsection{Établissement des Relations}

\begin{enumerate}[label=\arabic*.]
    \item Retournez dans la vue \textbf{Modèle} de Power BI
    \item Établissez les relations suivantes :
    \begin{itemize}
        \item \texttt{TF\_Commande[id\_temps] → Dim\_Temps[id\_temps]}
        \item \texttt{TF\_Commande[id\_seqEmployee] → Dim\_Employee[id\_seqEmployee]}
        \item \texttt{TF\_Commande[id\_seqClient] → Dim\_Client[id\_seqClient]}
    \end{itemize}
    \item Vérifiez que toutes les relations sont \textbf{1 à plusieurs (*)}
\end{enumerate}

\begin{figure}[H]
    \centering
    \includegraphics[width=1\linewidth]{Modèle de données final avec relations.png}
    \caption{Modèle de données final avec relations}
    \label{fig:data-model}
\end{figure}

\section{Comparaison Power BI vs Talend pour l'ETL}

\subsection{Tableau Comparatif}

\begin{table}[H]
\centering
\begin{tabular}{|p{0.25\textwidth}|p{0.35\textwidth}|p{0.35\textwidth}|}
\hline
\textbf{Aspect} & \textbf{Power BI (Power Query)} & \textbf{Talend} \\
\hline
\textbf{Type d'outil} & Outil de BI avec ETL intégré & Plateforme ETL/ELT dédiée \\
\hline
\textbf{Complexité} & Courbe d'apprentissage douce & Courbe d'apprentissage plus raide \\
\hline
\textbf{Langage} & M (Power Query) & Java + Composants visuels \\
\hline
\textbf{Connexions} & Connecteurs natifs limités & 1000+ connecteurs \\
\hline
\textbf{Transformation} & Interface utilisateur + M & Interface graphique + code \\
\hline
\textbf{Orchestration} & Basic (rafraîchissements) & Avancée (workflows complexes) \\
\hline
\textbf{Coût} & Gratuit (Desktop) & Licence payante (Enterprise) \\
\hline
\textbf{Performance} & Optimisé pour données de taille moyenne & Scalable pour gros volumes \\
\hline
\textbf{Maintenance} & Facile (tout intégré) & Complexe (infrastructure séparée) \\
\hline
\textbf{Utilisation projet Northwind} & Parfait pour POC et démonstrations & Surdimensionné pour ce besoin \\
\hline
\end{tabular}
\caption{Comparaison Power BI vs Talend pour l'ETL}
\end{table}

\subsection{Points Forts par Outil}

\subsubsection{Power BI (Power Query)}
\begin{itemize}
    \item ✅ Intégration native avec les rapports
    \item ✅ Interface utilisateur intuitive
    \item ✅ Pas de contexte switching entre ETL et reporting
    \item ✅ Version Desktop gratuite
    \item ✅ Fonctionnalités de modélisation intégrées
    \item ✅ Support DAX pour calculs avancés
\end{itemize}

\subsubsection{Talend}
\begin{itemize}
    \item ✅ Orchestration de workflows complexes
    \item ✅ Support de gros volumes de données
    \item ✅ Connecteurs très variés
    \item ✅ Qualité et gouvernance des données
    \item ✅ Métadonnées et documentation
    \item ✅ Intégration avec l'écosystème Big Data
\end{itemize}

\subsection{Exemple de Code Comparé}

\begin{table}[H]
\centering
\begin{tabular}{|p{0.45\textwidth}|p{0.45\textwidth}|}
\hline
\textbf{Power Query (M)} & \textbf{Talend (Java)} \\
\hline
\begin{lstlisting}[language=PowerQuery]
let
    Source = Orders_ssms,
    #"Filtered Rows" = 
        Table.SelectRows(
            Source, 
            each [ShippedDate] <> null
        )
in
    #"Filtered Rows"
\end{lstlisting} &
\begin{lstlisting}[language=Java]
// Composant tFilterRow
row1.ShippedDate != null

// Ou en Java dans tJavaRow
if(input_row.ShippedDate != null) {
    output_row = input_row;
}
\end{lstlisting} \\
\hline
\end{tabular}
\caption{Comparaison de syntaxe pour un filtre simple}
\end{table}

\subsection{Recommandation pour le Projet Northwind}

\begin{tcolorbox}[colback=yellow!5!white, colframe=yellow!75!black, title=⚠️ Choix Optimal pour Northwind]
\textbf{Power BI avec Power Query est le meilleur choix} pour ce projet car :
\begin{itemize}
    \item Volume de données modeste (878 lignes)
    \item Besoin d'intégration directe avec les rapports
    \item Temps de développement réduit
    \item Aucune infrastructure supplémentaire nécessaire
    \item Coût nul avec Power BI Desktop gratuit
    \item Facilité de maintenance et de partage
\end{itemize}
\end{tcolorbox}

\section{Expérience Pratique : Talend vs Power BI pour le projet Northwind}

\subsection{Contexte d'Implémentation}

Dans le cadre de ce projet, j'ai implémenté la même solution d'entrepôt de données Northwind avec deux outils différents : Power BI et Talend. L'objectif était de remplir un entrepôt de données vide sur SQL Server Management Studio (SSMS) à partir de deux sources distinctes :

\begin{itemize}
    \item \textbf{Source 1} : Base de données Northwind existante sur SSMS
    \item \textbf{Source 2} : Données complémentaires au format Microsoft Access
    \item \textbf{Cible} : Entrepôt SSMS vide avec modèle en étoile
\end{itemize}
\begin{figure}[H]
    \centering
    \includegraphics[width=0.9\linewidth]{Dim_Client_Talend.png}
    \caption{Job Talend complet pour la dimension client}
    \label{fig:talend-job-full}
\end{figure}
\begin{figure}[H]
    \centering
    \includegraphics[width=0.9\linewidth]{Dim_Employee_Talend.png}
    \caption{Job Talend complet pour la dimension employee}
    \label{fig:talend-job-full}
\end{figure}
\begin{figure}[H]
    \centering
    \includegraphics[width=0.9\linewidth]{Dim_Temps_Talend.png}
    \caption{Job Talend complet pour la dimension temps}
    \label{fig:talend-job-full}
\end{figure}
\begin{figure}[H]
    \centering
    \includegraphics[width=0.9\linewidth]{TF_Commande_Talend.png}
    \caption{Job Talend complet pour la table de fait}
    \label{fig:talend-job-full}
\end{figure}
\begin{figure}[H]
    \centering
    \includegraphics[width=0.9\linewidth]{TF_Commande_Jointure_Talend.png}
    \caption{Jointure de creation de la table de fait}
    \label{fig:talend-job-full}
\end{figure}

\subsection{Implémentation avec Talend}

\subsubsection{Architecture du Job Talend}

Le job Talend a été structuré comme suit :

\begin{enumerate}[label=\arabic*.]
    \item \textbf{Extraction des données sources} :
    \begin{itemize}
        \item Connexion à SQL Server via \texttt{tMSSqlInput}
        \item Lecture des fichiers Access via \texttt{tFileInputDelimited}
    \end{itemize}
    
    \item \textbf{Transformations principales} :
    \begin{itemize}
        \item \texttt{tMap\_2} : Jointures et transformations complexes
        \item \texttt{tMap} : Nettoyage et standardisation
        \item \texttt{tJavaRow} : Calculs personnalisés en Java
    \end{itemize}
    
    \item \textbf{Chargement vers SSMS} :
    \begin{itemize}
        \item \texttt{tMSSqlOutput} pour chaque table dimensionnelle
        \item Gestion des clés primaires et étrangères
    \end{itemize}
\end{enumerate}

\begin{figure}[H]
    \centering
    \begin{tikzpicture}[node distance=1cm]
    
    \tikzset{
        component/.style={
            draw,
            rectangle,
            rounded corners=5pt,
            align=center,
            minimum height=1cm,
            minimum width=2.5cm,
            fill=blue!10,
            line width=1pt
        },
        source/.style={
            draw,
            ellipse,
            align=center,
            minimum height=1cm,
            minimum width=2cm,
            fill=green!10,
            line width=1pt
        },
        target/.style={
            draw,
            ellipse,
            align=center,
            minimum height=1cm,
            minimum width=2cm,
            fill=red!10,
            line width=1pt
        },
        arrow/.style={
            ->,
            >=stealth,
            line width=1pt
        }
    }
    
    % Sources
    \node[source] (ssms) {SSMS\\Northwind};
    \node[source, right=2cm of ssms] (access) {Microsoft\\Access};
    
    % Composants d'entrée
    \node[component, below=1.5cm of ssms] (input1) {tMSSqlInput};
    \node[component, below=1.5cm of access] (input2) {tFileInput\\Delimited};
    
    % Composants de transformation
    \node[component, below=1.5cm of $(input1.south)!0.5!(input2.south)$] (tmap2) {tMap\_2};
    \node[component, left=1.5cm of tmap2] (tmap) {tMap};
    \node[component, right=1.5cm of tmap2] (javarow) {tJavaRow};
    
    % Sorties
    \node[target, below=2cm of tmap2] (tf) {TF\_Commande};
    \node[target, left=1cm of tf] (dimemp) {Dim\_Employee};
    \node[target, left=1cm of dimemp] (dimcli) {Dim\_Client};
    \node[target, right=1cm of tf] (dimtmp) {Dim\_Temps};
    
    % Flèches
    \draw[arrow] (ssms) -- (input1);
    \draw[arrow] (access) -- (input2);
    \draw[arrow] (input1) -- (tmap2);
    \draw[arrow] (input2) -- (tmap2);
    \draw[arrow] (tmap) -- (tmap2);
    \draw[arrow] (tmap2) -- (javarow);
    \draw[arrow] (javarow) -- (tf);
    \draw[arrow] (tmap) -- (dimemp);
    \draw[arrow] (tmap) -- (dimcli);
    \draw[arrow] (javarow) -- (dimtmp);
    
    \end{tikzpicture}
    \caption{Architecture simplifiée du flux de données Talend}
    \label{fig:talend-architecture}
\end{figure}

\subsection{Comparaison Pratique des Résultats}

\subsubsection{Volumes de Données Traités}

\begin{table}[H]
\centering
\begin{tabular}{|l|c|c|c|}
\hline
\textbf{Table} & \textbf{Lignes Sources} & \textbf{Talend} & \textbf{Power BI} \\
\hline
Dim\_Temps & 878 dates & 878 & 878 \\
Dim\_Client & 120 clients & 120 & 120 \\
Dim\_Employee & 58 employés & 58 & 58 \\
TF\_Commande & 878 commandes & 878 & 878 \\
\hline

\hline
\end{tabular}
\caption{Volumes identiques traités par les deux outils}
\end{table}

\subsubsection{Métriques Comparatives}

\begin{table}[H]
\centering
\begin{tabular}{|l|c|c|}
\hline
\textbf{Métrique} & \textbf{Talend} & \textbf{Power BI} \\
\hline
Temps de développement & 8-10 heures & 2-3 heures \\
Temps d'exécution & 15-20 secondes & 8-12 secondes \\
Complexité & Élevée (Java + XML) & Moyenne (M) \\
Apprentissage requis & 2-3 semaines & 2-3 jours \\
Maintenance & Infrastructure lourde & Fichier unique \\
Déploiement & Serveur requis & Fichier .pbix \\
\hline
\end{tabular}
\caption{Comparaison pratique des deux approches}
\end{table}

\subsection{Leçons Apprises et Recommandations}

\subsubsection{Pour les POC et Démonstrations}

\begin{tcolorbox}[colback=blue!5!white, colframe=blue!75!black, title=�� Power BI Recommandé]
\begin{itemize}
    \item \textbf{Vitesse} : Développement 3x plus rapide
    \item \textbf{Simplicité} : Pas d'infrastructure supplémentaire
    \item \textbf{Intégration} : ETL + modélisation + visualisation intégrés
    \item \textbf{Collaboration} : Partage facile avec Power BI Service
\end{itemize}
\end{tcolorbox}

\subsubsection{Pour les Environnements de Production}

\begin{tcolorbox}[colback=green!5!white, colframe=green!75!black, title=�� Talend Recommandé]
\begin{itemize}
    \item \textbf{Robustesse} : Gestion avancée des erreurs
    \item \textbf{Monitoring} : Suivi détaillé des exécutions
    \item \textbf{Scalabilité} : Traitement des gros volumes
    \item \textbf{Orchestration} : Workflows complexes et planifiés
\end{itemize}
\end{tcolorbox}

\subsection{Conclusion de l'Expérience}

Cette expérience pratique démontre que le choix entre Talend et Power BI dépend du contexte :

\begin{itemize}
    \item \textbf{Power BI} excelle pour la rapidité, la simplicité et l'intégration BI complète
    \item \textbf{Talend} est supérieur pour la robustesse, le monitoring et les environnements de production
    
    \item Pour le projet Northwind (1,456 lignes), Power BI était suffisant et plus rapide
    \item Pour des volumes plus importants ou des besoins de production, Talend serait nécessaire
\end{itemize}

\begin{tcolorbox}[colback=yellow!5!white, colframe=yellow!75!black, title=⚠️ Recommandation Finale]
\textbf{Utilisez Power BI pour} : POC, analyses ad-hoc, petites volumétries, équipes métier\\
\textbf{Utilisez Talend pour} : Pipelines de production, gros volumes, environnements réglementés, équipes techniques
\end{tcolorbox}

\section{Visualisation avec Python dans Power BI}

\subsection{Configuration de Python pour Power BI}

\subsubsection{Installation des Prérequis}
\begin{enumerate}[label=\arabic*.]
    \item \textbf{Installer Python} : Téléchargez Python 3.8+ depuis \url{https://python.org}
    \item \textbf{Installer les packages} :
    \begin{lstlisting}[language=bash]
pip install pandas matplotlib numpy seaborn
    \end{lstlisting}
    \item \textbf{Configurer Power BI} :
    \begin{itemize}
        \item Fichier → Options et paramètres → Options
        \item Scripting Python → Spécifier le chemin d'installation Python
        \item Exemple : \texttt{C:\textbackslash Users\textbackslash IHADDADENE Chakib\textbackslash AppData\textbackslash Local\textbackslash Programs\textbackslash Python\textbackslash Python39}
    \end{itemize}
\end{enumerate}

\subsubsection{Création d'un Visuel Python}
\begin{enumerate}[label=\arabic*.]
    \item \textbf{Sélectionner le visuel Python} :
    \begin{itemize}
        \item Dans le volet Visualisations, cliquez sur l'icône \textbf{Python}
        \item Une zone de script vide apparaît sur la page
    \end{itemize}
    \begin{figure}[H]
        \centering
        \includegraphics[width=0.25\linewidth]{visuel python.png}
        \caption{Selection de l'icone PY}
        \label{fig:placeholder}
    \end{figure}
    \item \textbf{Glisser les champs nécessaires} :
    \begin{itemize}
        \item Depuis le volet Données, glissez-déposez les colonnes dans la partie Valeurs
        \item Exemple pour notre projet Northwind :
        \begin{itemize}
            \item Depuis \texttt{TF\_Commande} : \texttt{nbr\_commande\_livrees}
            \item Depuis \texttt{Dim\_Employee} : \texttt{TerritoryDescri}
        \end{itemize}
    \end{itemize}
    \begin{figure}[H]
        \centering
        \includegraphics[width=0.3\linewidth]{Importation des colonnes.png}
        \caption{Glissement des champs nécessaires}
        \label{fig:placeholder}
    \end{figure}
    
    \item \textbf{Écrire le script Python} :
    \begin{itemize}
        \item Le champ de script s'active automatiquement
        \item Power BI crée automatiquement un DataFrame \texttt{dataset} contenant vos données
    \end{itemize}
\end{enumerate}

\begin{figure}[H]
    \centering
    \includegraphics[width=1\linewidth]{script_python.png}
    \caption{Interface du visuel Python dans Power BI}
    \label{fig:python-visual-interface}
\end{figure}

\subsection{Les 6 Graphiques Python Implémentés}

\subsubsection{Graphique 1 : Volume des Commandes par Mois}

\begin{lstlisting}[language=Python, caption=Volume des commandes livrées et non livrées par mois]
import pandas as pd
import matplotlib.pyplot as plt

# Vérification des colonnes
colonnes_requises = ['id_temps', 'nbr_commande_livrees', 'nbr_commande_non_livrees', 'mois_annee']
colonnes_manquantes = [col for col in colonnes_requises if col not in dataset.columns]

if colonnes_manquantes:
    fig, ax = plt.subplots(figsize=(10, 2))
    ax.text(0.5, 0.5, f"Glissez ces colonnes:\n{', '.join(colonnes_manquantes)}",
            ha='center', va='center', fontsize=12, color='red',
            bbox=dict(boxstyle="round,pad=0.5", facecolor="yellow", alpha=0.7))
    ax.axis('off')
    plt.show()

else:
    # Agrégation par mois
    commandes_par_mois = dataset.groupby('mois_annee').agg({
        'nbr_commande_livrees': 'sum',
        'nbr_commande_non_livrees': 'sum'
    }).reset_index()

    commandes_par_mois['total_commandes'] = (
        commandes_par_mois['nbr_commande_livrees'] +
        commandes_par_mois['nbr_commande_non_livrees']
    )

    # Trier par date
    try:
        commandes_par_mois['date_sort'] = pd.to_datetime(
            commandes_par_mois['mois_annee'] + '/01',
            format='%m/%Y/%d'
        )
        commandes_par_mois = commandes_par_mois.sort_values('date_sort')
    except:
        commandes_par_mois = commandes_par_mois.sort_values('mois_annee')

    # Création du graphique
    fig, ax = plt.subplots(figsize=(14, 7))

    x = range(len(commandes_par_mois))
    mois_labels = commandes_par_mois['mois_annee'].tolist()

    # Barres empilées
    ax.bar(x, commandes_par_mois['nbr_commande_livrees'],
           label='Commandes Livrées', color='green', alpha=0.7)
    ax.bar(x, commandes_par_mois['nbr_commande_non_livrees'],
           bottom=commandes_par_mois['nbr_commande_livrees'],
           label='Commandes Non Livrées', color='red', alpha=0.7)

    ax.set_xlabel('Mois', fontsize=12)
    ax.set_ylabel('Nombre de Commandes', fontsize=12)
    ax.set_title(' VOLUME DES COMMANDES PAR MOIS', fontsize=14, fontweight='bold')
    ax.set_xticks(x)
    ax.set_xticklabels(mois_labels, rotation=45, ha='right')
    ax.legend()
    ax.grid(True, alpha=0.3)

    # Ajouter les totaux au-dessus des barres
    for i, total in enumerate(commandes_par_mois['total_commandes']):
        ax.text(i, total + (total*0.01), f'{total:,}',
                ha='center', va='bottom', fontsize=9)

    # Statistiques
    total_periode = commandes_par_mois['total_commandes'].sum()
    stats_text = f"Total période: {total_periode:,} commandes"

    plt.figtext(0.02, 0.02, stats_text, fontsize=10,
                bbox=dict(boxstyle="round,pad=0.5", facecolor="lightgray", alpha=0.8))

    plt.tight_layout()
    plt.show()
\end{lstlisting}

\begin{figure}[H]
    \centering
    \includegraphics[width=0.9\linewidth]{figure1.png}
    \caption{Volume des commandes livrées et non livrées par mois}
    \label{fig:graph1}
\end{figure}

\subsubsection{Graphique 2 : Top 10 Clients par Commandes Livrées}

\begin{lstlisting}[language=Python, caption=Top 10 clients par nombre de commandes livrées]
import pandas as pd
import matplotlib.pyplot as plt

# Grouper par client et sommer les commandes livrées
df = dataset.groupby('CompanyName')['nbr_commande_livrees'].sum().nlargest(10)

plt.figure(figsize=(10, 6))
df.sort_values().plot(kind='barh', color='skyblue')
plt.title("Top 10 Clients par commandes livrées", fontsize=14, fontweight='bold')
plt.xlabel("Nombre de commandes livrées", fontsize=12)
plt.grid(axis='x', alpha=0.3)

# Ajouter les valeurs sur les barres
for i, v in enumerate(df.sort_values()):
    plt.text(v + (v*0.01), i, f'{v:,}', va='center', fontsize=10)

plt.tight_layout()
plt.show()
\end{lstlisting}

\begin{figure}[H]
    \centering
    \includegraphics[width=1\linewidth]{figure2.png}
    \caption{Top 10 clients par nombre de commandes livrées}
    \label{fig:graph2}
\end{figure}

\subsubsection{Graphique 3 : Top 5 Territoires par Commandes Livrées}

\begin{lstlisting}[language=Python, caption=Top 5 territoires par performance]
import pandas as pd
import matplotlib.pyplot as plt

df = dataset.groupby('TerritoryDescri')['nbr_commande_livrees'].sum().nlargest(5)

plt.figure(figsize=(10, 6))
df.sort_values().plot(kind='barh', color='purple')
plt.title("Top 5 Territoires par commandes livrées", fontsize=14, fontweight='bold')
plt.xlabel("Nombre de commandes", fontsize=12)
plt.grid(axis='x', alpha=0.3)

# Ajouter les valeurs sur les barres
for i, v in enumerate(df.sort_values()):
    plt.text(v + (v*0.01), i, f'{v:,}', va='center', fontsize=10, color='white')

plt.tight_layout()
plt.show()
\end{lstlisting}

\begin{figure}[H]
    \centering
    \includegraphics[width=1\linewidth]{figure3.png}
    \caption{Top 5 territoires par nombre de commandes livrées}
    \label{fig:graph3}
\end{figure}

\subsubsection{Graphique 4 : Heatmap Clients vs Employés}

\begin{lstlisting}[language=Python, caption=Heatmap des commandes livrées par client et employé]
import pandas as pd
import matplotlib.pyplot as plt
import numpy as np

# Créer nom complet de l'employé
dataset['Employe'] = dataset['Nom'] + " " + dataset['Prenom']

# Calcul total commandes par client
total_client = dataset.groupby('CompanyName')['nbr_commande_livrees'].sum()

# Top 20 clients
top_clients = total_client.nlargest(20).index
df_top = dataset[dataset['CompanyName'].isin(top_clients)]

# Pivot table : clients en lignes, employés en colonnes
df_pivot = df_top.pivot_table(index='CompanyName', columns='Employe', 
                              values='nbr_commande_livrees', aggfunc='sum', fill_value=0)

plt.figure(figsize=(14, 8))
plt.imshow(df_pivot, cmap='YlGnBu', aspect='auto', interpolation='nearest')
plt.colorbar(label='Commandes livrées', shrink=0.8)
plt.xticks(range(len(df_pivot.columns)), df_pivot.columns, rotation=90, fontsize=9)
plt.yticks(range(len(df_pivot.index)), df_pivot.index, fontsize=9)
plt.title("Heatmap : Top 20 Clients x Employés", fontsize=14, fontweight='bold')

# Ajouter les valeurs dans les cellules
for i in range(len(df_pivot.index)):
    for j in range(len(df_pivot.columns)):
        value = df_pivot.iloc[i, j]
        if value > 0:
            plt.text(j, i, f'{int(value)}', ha='center', va='center', 
                    color='black' if value < df_pivot.values.max()/2 else 'white',
                    fontsize=8)

plt.tight_layout()
plt.show()
\end{lstlisting}

\begin{figure}[H]
    \centering
    \includegraphics[width=1\linewidth]{figure4.png}
    \caption{Heatmap des commandes livrées par client et employé}
    \label{fig:graph4}
\end{figure}

\subsubsection{Graphique 5 : Répartition par Région}

\begin{lstlisting}[language=Python, caption=Répartition des commandes livrées par région]
import pandas as pd
import matplotlib.pyplot as plt

df = dataset.groupby('Region')['nbr_commande_livrees'].sum()

plt.figure(figsize=(9, 9))
wedges, texts, autotexts = plt.pie(df.values, labels=df.index, autopct='%1.1f%%', 
                                   colors=plt.cm.Set3.colors, startangle=90,
                                   wedgeprops=dict(edgecolor='white', linewidth=2))

# Améliorer l'apparence des pourcentages
for autotext in autotexts:
    autotext.set_color('black')
    autotext.set_fontsize(10)
    autotext.set_fontweight('bold')

plt.title("Répartition des commandes livrées par région", fontsize=14, fontweight='bold')

# Ajouter une légende
plt.legend(wedges, df.index, title="Régions", loc="center left", 
           bbox_to_anchor=(1, 0, 0.5, 1), fontsize=10)

plt.tight_layout()
plt.show()
\end{lstlisting}

\begin{figure}[H]
    \centering
    \includegraphics[width=0.8\linewidth]{figure5.png}
    \caption{Répartition des commandes livrées par région}
    \label{fig:graph5}
\end{figure}

\subsubsection{Graphique 6 : Évolution Mensuelle des Commandes}

\begin{lstlisting}[language=Python, caption=Évolution des commandes livrées et non livrées]
import pandas as pd
import matplotlib.pyplot as plt

# Calculer le total des commandes par mois
df = dataset.groupby('mois_annee')[['nbr_commande_livrees','nbr_commande_non_livrees']].sum()
df = df.sort_index()  # trier par mois

plt.figure(figsize=(12, 7))
plt.plot(df.index, df['nbr_commande_livrees'], marker='o', linestyle='-', 
         color='green', label='Livrées', linewidth=2, markersize=8)
plt.plot(df.index, df['nbr_commande_non_livrees'], marker='s', linestyle='--', 
         color='red', label='Non-livrées', linewidth=2, markersize=8)

plt.title("Évolution mensuelle des commandes livrées et non-livrées", 
          fontsize=14, fontweight='bold')
plt.xlabel("Mois", fontsize=12)
plt.ylabel("Nombre de commandes", fontsize=12)
plt.xticks(rotation=45, ha='right')
plt.legend(fontsize=12)
plt.grid(True, alpha=0.3)

# Remplir l'aire entre les courbes
plt.fill_between(df.index, df['nbr_commande_livrees'], alpha=0.2, color='green')
plt.fill_between(df.index, df['nbr_commande_non_livrees'], alpha=0.2, color='red')

# Ajouter les valeurs sur les points
for i, (liv, nliv) in enumerate(zip(df['nbr_commande_livrees'], df['nbr_commande_non_livrees'])):
    plt.text(i, liv + (liv*0.05), f'{liv:,}', ha='center', va='bottom', fontsize=9)
    plt.text(i, nliv - (nliv*0.05), f'{nliv:,}', ha='center', va='top', fontsize=9)

plt.tight_layout()
plt.show()
\end{lstlisting}

\begin{figure}[H]
    \centering
    \includegraphics[width=0.9\linewidth]{figure6.png}
    \caption{Évolution mensuelle des commandes livrées et non livrées}
    \label{fig:graph6}
\end{figure}

\subsection{Explications des Codes Python}

\subsubsection{Structure Commune des Scripts}

Chaque script Python suit la même structure logique :

\begin{enumerate}[label=\arabic*.]
    \item \textbf{Importation des bibliothèques} : pandas pour la manipulation des données, matplotlib pour la visualisation
    \item \textbf{Vérification des données} : Validation des colonnes nécessaires dans le DataFrame
    \item \textbf{Agrégation des données} : Groupement et calcul des métriques
    \item \textbf{Création du graphique} : Configuration du type, des couleurs et des labels
    \item \textbf{Personnalisation} : Ajout de valeurs, de grilles et de légendes
    \item \textbf{Affichage} : Utilisation de \texttt{plt.show()} pour afficher dans Power BI
\end{enumerate}

\subsubsection{Techniques Avancées Utilisées}

\begin{table}[H]
\centering
\begin{tabular}{|p{0.3\textwidth}|p{0.7\textwidth}|}
\hline
\textbf{Technique} & \textbf{Application dans les scripts} \\
\hline
\textbf{Groupby + Agg} & Agrégation des données par différentes dimensions (mois, client, territoire) \\
\textbf{Pivot Table} & Création de heatmaps avec des matrices clients × employés \\
\textbf{Data Validation} & Vérification des colonnes nécessaires avant le traitement \\
\textbf{Visual Customization} & Personnalisation fine des couleurs, tailles et polices \\
\textbf{Annotations} & Ajout de valeurs numériques directement sur les graphiques \\
\textbf{Error Handling} & Gestion des erreurs de format de date avec try-except \\
\textbf{Color Mapping} & Utilisation de palettes de couleurs professionnelles (Set3, YlGnBu) \\
\hline
\end{tabular}
\caption{Techniques Python avancées utilisées dans les visualisations}
\end{table}

\subsection{Bonnes Pratiques pour les Visuals Python}

\begin{table}[H]
\centering
\begin{tabular}{|p{0.4\textwidth}|p{0.55\textwidth}|}
\hline
\textbf{Pratique} & \textbf{Explication} \\
\hline
Vérification des données & Vérifier les colonnes nécessaires avant le traitement \\
Gestion des erreurs & Utiliser try-except pour les conversions de date \\
Optimisation des performances & Limiter le nombre de lignes avec des filtres \\
Personnalisation visuelle & Utiliser des palettes de couleurs professionnelles \\
Annotations & Ajouter des valeurs pour améliorer la lisibilité \\
Titres descriptifs & Utiliser des titres clairs et informatifs \\
Grilles légères & Ajouter des grilles pour faciliter la lecture \\
Export des scripts & Sauvegarder les scripts dans des fichiers .py externes \\
\hline
\end{tabular}
\caption{Bonnes pratiques pour les visuals Python dans Power BI}
\end{table}

\subsection{Avantages du Visuel Python}

\begin{itemize}
    \item \textbf{✅ Flexibilité totale} : Accès à toutes les bibliothèques Python (pandas, matplotlib, seaborn)
    \item \textbf{✅ Personnalisation avancée} : Contrôle complet sur chaque élément du graphique
    \item \textbf{✅ Interactivité} : Réaction aux filtres Power BI en temps réel
    \item \textbf{✅ Intégration native} : Pas besoin d'exporter/importer les données
    \item \textbf{✅ Reproducibilité} : Scripts reproductibles et versionnables
    \item \textbf{✅ Analyses complexes} : Possibilité d'implémenter des analyses statistiques avancées
\end{itemize}

\begin{tcolorbox}[colback=yellow!5!white, colframe=yellow!75!black, title=⚠️ Limitations à connaître]
\begin{itemize}
    \item Performance : Les scripts complexes peuvent ralentir Power BI
    \item Mémoire : Limité par les ressources de Power BI Desktop
    \item Dépendances : Packages à installer sur chaque machine
    \item Version Python : Doit correspondre à celle configurée dans Power BI
    \item Courbe d'apprentissage : Requiert des compétences en Python
    \item Debugging : Plus difficile que les visuels natifs de Power BI
\end{itemize}
\end{tcolorbox}


\newpage
\subsection{Dashboard Final Northwind}

Notre dashboard final intègre 6 visuals Python interconnectés, permettant une analyse complète des données Northwind :

\begin{figure}[H]
    \centering
    \includegraphics[width=1\linewidth]{dashboard.png}
    \caption{Dashboard Northwind avec 6 visuals Python intégrés}
    \label{fig:final-dashboard}
\end{figure}

\section{Résolution des Problèmes Courants}

\subsection{Problèmes d'Installation}

\begin{table}[H]
\centering
\begin{tabular}{|p{0.3\textwidth}|p{0.6\textwidth}|}
\hline
\textbf{Problème} & \textbf{Solution} \\
\hline
Échec d'installation Microsoft Store & Téléchargez la version hors Store \\
Erreur "Accès refusé" & Exécutez en tant qu'administrateur \\
Manque de dépendances & Installez .NET Framework 4.8 \\
Antivirus bloque l'installation & Désactivez temporairement l'antivirus \\
\hline
\end{tabular}
\end{table}

\subsection{Problèmes de Connexion}

\begin{table}[H]
\centering
\begin{tabular}{|p{0.3\textwidth}|p{0.6\textwidth}|}
\hline
\textbf{Problème} & \textbf{Solution} \\
\hline
SQL Server inaccessible & Vérifiez le nom du serveur et les permissions \\
Fichier Excel non trouvé & Vérifiez le chemin et les permissions \\
Erreur d'authentification & Utilisez l'authentification Windows \\
Limite de mémoire & Augmentez la mémoire allouée à Power BI \\
\hline
\end{tabular}
\end{table}

\subsection{Problèmes avec Python}

\begin{table}[H]
\centering
\begin{tabular}{|p{0.3\textwidth}|p{0.6\textwidth}|}
\hline
\textbf{Problème} & \textbf{Solution} \\
\hline
Python non détecté & Vérifiez le chemin dans Options → Scripting Python \\
Packages manquants & Exécutez \texttt{pip install pandas matplotlib} \\
Erreurs d'import & Redémarrez Power BI après installation des packages \\
Graphiques vides & Vérifiez que les champs nécessaires sont glissés \\
Performance lente & Limitez le nombre de lignes avec des filtres \\
\hline
\end{tabular}
\caption{Problèmes courants avec Python dans Power BI}
\end{table}

\newpage
\section{Conclusion}

Ce guide vous a accompagné de l'installation de Power BI Desktop à la création d'un projet d'entrepôt de données complet avec Northwind, incluant une analyse comparative avec Talend et deux approches Python. Vous disposez maintenant de toutes les ressources nécessaires pour :

\begin{itemize}
    \item ✅ Installer et configurer Power BI Desktop
    \item ✅ Comprendre l'architecture d'un entrepôt de données en étoile
    \item ✅ Comparer Power BI et Talend pour l'ETL
    \item ✅ Créer un projet structuré Northwind
    \item ✅ Importer et transformer des données multi-sources
    \item ✅ Construire un modèle de données optimisé
    \item ✅ Utiliser Python dans Power BI pour des visualisations avancées
    \item ✅ Développer des dashboards Python autonomes sans Power BI
    \item ✅ Implémenter les 6 graphiques d'analyse Northwind
    \item ✅ Choisir l'approche adaptée à vos besoins
    \item ✅ Résoudre les problèmes courants
    \item ✅ Publier et partager vos analyses
\end{itemize}

\begin{tcolorbox}[colback=green!5!white, colframe=green!75!black, title=�� Compétences Acquises]
À travers ce projet, vous avez développé des compétences précieuses en :
\begin{itemize}
    \item \textbf{ETL} : Transformation de données avec Power Query et Talend
    \item \textbf{Modélisation} : Conception de modèle en étoile
    \item \textbf{Visualisation} : Création de dashboards interactifs
    \item \textbf{Programmation} : Scripting Python avancé
    \item \textbf{Analyse} : Interprétation de données business
    \item \textbf{Comparaison} : Évaluation d'outils BI/ETL
\end{itemize}
\end{tcolorbox}

\textbf{Perspectives futures} : Cette base solide vous permettra d'aborder des projets plus complexes, d'intégrer des sources de données supplémentaires, et d'explorer des techniques avancées comme le machine learning intégré ou le streaming de données en temps réel.

\end{document}